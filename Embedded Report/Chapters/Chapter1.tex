% Chapter 1

\chapter{Introduction} % Write in your own chapter title
\label{Chapter1}
\lhead{Chapter 1. \emph{Introduction}} % Write in your own chapter title to set the page header

The increasing scale and complexity of the buildings
 has brought major challenges to fire safety with 
 rapid economic development. Fast fire detection 
 and warning with high sensitivity and precision 
 is therefore necessary to raising fire losses. 
 Nevertheless, conventional fire alarm systems, 
 such as smoke and heat detectors, are not ideal 
 for wide cities, complicated structures or spaces 
 with multiple disruptions. Due to the limitations 
 of the above detection technologies, missing 
 detections, false alarms, delays in detection 
 and other problems often occur, making early warnings 
 even more difficult to achieve.

 Image fire detection has recently become a hot research topic. 
 The technology provides many advantages, including early fire 
 detection, high precise deployment of the device and the 
 ability in large space and complex building systems to 
 effectively detect fire. In order for images to be present, 
 they process image data from a camera through algorithms. 
 The detection algorithm is therefore at the very heart of 
 this technology that decide directly the fire detector output.

 The framework of fire detection algorithms includes three 
 main phases, including image preprocessing, extraction 
 and detection of fires. The central part of algorithms 
 is among the feature extraction. The conventional algorithm 
 relies on the fire and machine learning classification manually 
 picked. It is the algorithms weakness to rely on professional 
 knowledge for manual selection of features.Although the 
 researchers develop many studies on image characteristics 
 of smoke and fire, only simple image characteristics, 
 like colour. However, the extraction of small and medium-sized 
 complex image features is difficult to differentiate between 
 fire and fire, causing a lower accuracy and poor 
 generalization efficiency, thanks to diverse fire forms and 
 scenes as well as many interference events in the field.

 Current neural network (CNN's) image recognition algorithms 
 can learn and extract complex image functions automatically. 
 This type of algorithm has caused serious concern with high 
 visual search efficiency, automatic driving, medical diagnoses, 
 etc. Some scientists therefore apply CNNs to fire detection and 
 improve the self-learned algorithm into fire image array.


 Although CNN-based fire detection algorithms are more prominent 
 than traditional algorithms in complex situations in detection 
 accuracy, there are still some issues. First, existing 
 machine-learning algorithms were mostly categorized as 
 image fire detection and the stage proposed for the area was 
 ignored. The algorithms separate the entire picture into a 
 single class. Yet smoke and flame only filled a small area 
 of the picture at the early stage of fire. If smoke and 
 inflammation are not apparent, the whole picture feature will 
 decrease the accuracy of detection and delay fire detection 
 and alarm activation without regional suggestions.Therefore, 
 before image classification proposals regions should be defined 
 to enhance the algorithm's capacity to detect early fire. 
 Secondly, other scholars developed proposed regions through 
 the manual selection of features and the classification 
 of CNNs for pro-posal regions. This type of algorithm, which 
 generates the proposed regions individually by computing, 
 does not use CNNs as part of the global detection process , 
 resulting in large numbers and a slow rate of detection.
 In this project we implement CNNs based fire detections model to 
 detects the features.